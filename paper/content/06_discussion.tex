\section{Discussion}
\label{sec:discussion}

The results show that adaptive hybrid AI's contribution to competitive advantage is very pragmatic: the role of such systems 
is to reduce costs by reducing workloads, or by increasing efficiency by avoiding wrong allocation of resources. In some cases 
adaptive hybrid AI systems could help to improve product quality or customer experience. However, the findings suggest that 
the contribution of adaptive hybrid AI to competitive advantage is limited to incremental improvements. This somewhat contrasts
with previous findings reported in \citet{balakrishnanGlobalSurveyState2020} where increased revenues where attributed as a main
benefit of the application of AI in enterprises.
In two cases innovations based on an adaptive, hybrid AI may lead to increased revenues. Firstly, in desk review, we found that a
system which effectively distinguishes papers that are likely to be accepted may also contribute to increased revenues. Secondly,
one group of participants suggested that certain quality and background checks could be offered as paid-for add-on services.
Of course the findings are limited by the fact that we decomposed the editorial process in steps and investigated use cases
for each step. Thus, the ideas and scenarios that were developed represent incremental improvements and will not lead to 
transformative business models with the potential to disrupt the industry.

The nature of this research was exploratory and on a conceptual level. Thus, the findings are very much limited. To further
investigate the potential on enterprise competitiveness of adaptive hybrid AI systems in scholarly publishing, we suggest
conducting a more detailed research following the principles of Design Science Research (DSR) in information systems \citep{hevnerDesignScienceInformation2004,
vaishnaviDesignResearchInformation2004} and the Design Science Research Methodology (DSRM) in particular \citep{peffersDesignScienceResearch2007}. 
As part of a DSRM approach, an adaptive hybrid AI system could be developed as a proof-of-concept and evaluated in a real-world
setting. This would allow drawing more robust conclusions on competitive advantage arising out of such a system, especially about 
the types of different competitive advantages and their importance to overall competitiveness. However, according
to \citet{gudigantalaAIDecisionmakingFramework2023}, enterprises need to differentiate between soft and hard benefits arising out
of AI applications. Soft benefits are difficult to quantify and may be difficult to attribute to a specific AI application. This 
means that the true extent on competitive advantages of adaptive hybrid AI systems may be difficult to assess and quantify.
