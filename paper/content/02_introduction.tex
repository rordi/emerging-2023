\section{Introduction}
\label{sec:introduction}

As enterprises face universally accelerating change \citep{eliazarUniversalityAcceleratingChange2018}, it
is crucial for their artificial intelligence (AI) systems to be dynamic and adaptable. Adaptive and hybrid
intelligent (HI) systems that learn from new data, from the interaction with human agents and work
synergistically with humans can provide a competitive advantage to enterprises. This study explored the types
of competitive advantage arising for enterprises that adopt adaptive hybrid intelligent systems using the case
of one industry. It answers the following research question: \textit{What types of competitive advantages can
arise out of using adaptive, hybrid AI systems in the editorial process of academic publishers?} The paper
contributes to the literature by conceptually exploring the research question using the case of the scholarly
publishing industry. This could guide future research on competitive advantages of AI involving real-world
implementations of adaptive hybrid intelligent systems.

A mixed-methods approach was used, consisting of a literature review and qualitative data collection from
a focus group of graduate students in information systems ($n = 26$). The literature review identified how 
adaptive hybrid intelligent systems contribute to competitive advantage of enterprises. This was used to 
create a categorization of competitive advantages of AI. The focus group was split in 4 subgroups and each
presented with a scenario involving an adaptive hybrid intelligent system within the editorial process in
scholarly publishing. After being introduced to adaptive and hybrid intelligence and types of competitive 
advantages of AI, the subgroups were asked to discuss which type of competitive advantage was relevant in
the scenario using the categorization.

The findings of this study are fourfold. Firstly, as part of writing research manuscripts, we identified
``Improved performance'' and ``Improved customer experience'' as the two main types of competitive advantage.
Secondly, as part of finding appropriate journals and formatting a paper according to the journal's style, 
we identified ``Improved performance'' and ``Reduced costs'' as the relevant types of competitive advantage.
Thirdly, as part of the editor's desk review of manuscripts, we identified ``Reduced costs'' and ``Improved
customer experience'' as the relevant types of competitive advantage. Fourthly, as part of finding suitable 
peer-reviewers for a manuscript, we identified ``Reduced costs'', ``Improved performance'' and ``Better 
products \& services'' as the relevant types of competitive advantage. Overall, the findings suggest that
adaptive hybrid intelligent systems can provide a competitive advantage to enterprises in the scholarly
publishing industry. However, the advantages might be very pragmatic and mainly contributing in terms of
lowering costs, increasing process efficiencies and (marginally) improving product or service quality. Based
on the conceptual nature of this paper, more research is needed to validate the findings in a real-world
setting. The findings of this study can be used as a basis to guide further research in the area of
competitive advantages that result from the use of adaptive and hybrid intelligent information systems.

The remainder of the paper is structured as follows. Section~\ref{sec:literature} presents the theoretical
background on hybrid intelligent systems and the competitive advantage from AI for enterprises. Section~\ref{sec:methods}
describes the methodological approach of the study. Section~\ref{sec:results} presents the findings. The paper
concludes in Section~\ref{sec:discussion} with a discussion of the findings and limitations of the study.