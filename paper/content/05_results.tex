\section{Results}
\label{sec:results}

This section presents the results. Based on the findings from the literature review in Section \ref{sec:literature},
mostly based on \citet{iansitiCompetingAgeAI2020,hoArtificialIntelligenceFirm2022}, we define the following dimensions
for assessing the competitive advantage of AI systems:

\begin{enumerate}
    \item Reduced costs: AI streamlines processes and reduces the costs of performing processes.
    \item Improved performance: AI improves the performance of the processes.
    \item Better decision-making: AI helps with data-driven decision-making.
    \item Higher customer satisfaction: AI contributes to higher customer satisfaction.
    \item Better customer segmentation: AI helps to better segment and better target customers.
    \item Improved customer experience: AI improves the customer experience.
    \item Better products \& services: AI enables product and service innovations.
    \item Business innovation: AI enables new business models.
\end{enumerate}

\vspace{0.4cm}

\noindent The types of competitive advantages are subsequently used to assess the potential of adaptive hybrid AI systems.
Table~\ref{tab:editorialProcess} shows an overview of potential use cases for hybrid AI for each step in the typical editorial process
introduced previously in Section~\ref{sec:literature}. In the following we will focus on 4 areas of use cases, which correspond to the 
4 scenarios presented in Appendix \ref{appendix:scenarios}. The 4 use cases of AI are focused on the following topics: how can adaptive
and hybrid AI contribute to the competitive advantage of an enterprise by (1) facilitating the process of writing a manuscript; (2) helping
authors to find appropriate journals to submit to and format the paper according to the journal's style; (3) improving the desk
review process; and (4) improving the process of finding peer-reviewers. Each of these 4 domains is discussed in more detail in the
following subsections, including presenting the use cases, discussing the competitive advantage for a scholarly publisher of
implementing the use case, and contrasting with the qualitative feedback from the focus group participants.

\subsection{Writing a Manuscript}

\subsubsection*{Adaptive and Hybrid AI Use Cases}
Writing a scholarly manuscript involves several tasks, such as literature search, summarization of key findings, and writing-up the
literature review. This is a time-consuming task that requires a lot of effort from the author. Hybrid AI could help to automate aspects
of the literature search process. For example, AI could recommend relevant papers based on the authors' previous publications and search
history. Whenever the author reads a paper, the author could classify it according to some user-defined criteria, e.g., ``relevant'',
``not relevant'', ``method'', ``result'' and categorized by topics. Humans would thus steer the classification scheme, while the AI could
learn to adapt to the user by identifying relevant papers and classifying them according to the user's preferences. Such an adaptive AI
could automatically screen new recent literature to expand the author's collection while continuously collecting additional feedback from
the author to refine the recommendations. Finally, when the user writes its manuscript, the AI can recommend relevant papers from the
collection based on the context and narrative in the author's manuscript.

Further, the system could remind authors
on particular manuscript sections to include, the order of sections, and automatically formatting the manuscript according to the journal's
style. Some sections, such as funding, conflict of interest, and data availability statements, could be automatically adjusted to the journal's 
preferred wording. In case of doubt, the LLM could offer a chatbot interface to answer questions that the author may have about specific format
and style issues. The author could thus receive pertinent feedback based on the context of the manuscript and the journal it is being 
written for.

Another frequent problem in writing the manuscript are language issues. Many scholars are not native English speakers but have to
communicate their results in that language, especially in the natural and social sciences. An adaptive AI could help to improve the
writing and language quality of manuscripts by providing suggestions for (re)writing sentences. Because the author may have his own
preferred style of writing, the AI learns to adapt to the author's style and provide suggestions that are improving the clarity and
grammatical correctness of the text, while still following the author's style. 

\subsubsection*{Competitive Advantages} 
A writing assistant system based on a LLM that helps authors in writing their manuscript could constitute a competitive advantage for 
publishers as authors could write papers faster. Submission processes for journals could be integrated into the writing process, thereby
increasing the process efficiency. A recent study cited in \textit{Nature} estimated that time authors spent on formatting manuscripts
could be worth 230 million USD annually \citep{kozlovRevealedMillionsDollars2023,clotworthySavingTimeMoney2023}. A writing assistant
system could help to reduce this time and hence increase the efficiency of the writing process. This would also translate into a better
customer experience for the author. Using an LLM as a writing assistant could also streamline later peer-review and production processes.
As papers that will be submitted to the  publisher are in a clearer language, efficient peer-review and decision-making is facilitated.
This translates into faster turnaround times and better cost efficiency in the peer-review and subsequent production process (i.e.,
copy-editing, typesetting, proofreading).

\subsubsection*{Findings}
A group of 6 participants was presented with the first scenario of writing a manuscript and asked to discuss the competitive 
advantages of such a system based on adaptive hybrid AI. As part of their discussion, the participants mentioned better decision-making
(arguing in the perspective of an author): if the literature review is automated by an AI system that can adapt to the users' preferences, 
the system could help authors to make better decisions on which papers to cite. This would further free up time for authors
to focus on the actual content of the paper. This finding does not directly explain a competitive advantage for the
publisher. However, as the process is more effective for the author, the author may write more papers and hence submit 
more manuscripts to the publisher. We could therefore categorize this finding as ``Improved performance''. Similarly, the participants
mentioned that an AI system could act as a writing assistant for foreign language authors by offering automated translations, which 
they considered to be of type ``Improved customer experience''.



\subsection{Finding Appropriate Journals \& Formatting}

\subsubsection*{Adaptive and Hybrid AI Use Cases}

When an author has finished writing the manuscript, the next step is to find an appropriate journal to submit the manuscript to.
Authors may typically have some journals that they regularly read in mind to potentially submit their paper to. However, given that 
manuscripts are often rejected after peer-review, an AI system that adapts to the user's preferred journals---yet has the capability 
to predict which journal is most likely to accept the manuscript---could be a valuable tool for authors to reduce the time to publication.
The AI system could learn the author's preferred journals from its previous publications but also from journals typically cited by the author
in its previous paper. Additionally, the author could provide input data, such as if the journal needs to be covered in a specific
index or if the journal needs to be open access. Further, the AI system can learn patterns of acceptance and rejection of manuscripts
in different journals. While data on rejections are not publicly available, the AI system could learn from the author's own previous
submissions and rejections. The acceptance, however, can be learned from public data as the accepted manuscripts end up being published, 
i.e., constitute publicly available data. The AI system could then present a ranking of journals that match with the user's preferences
and are most likely to accept the manuscript.

\subsubsection*{Competitive Advantages} 

A publisher offering such a journal finder and ranking system to guide authors in finding appropriate journals could have a competitive
advantage due to  faster publication times and higher customer satisfaction. The faster publication time is due to lower overall 
probability of rejection, while higher satisfaction is due to fewer rejections and fast time to publication. The AI system could also
be used to recommend journals that are not yet on the author's radar. This could be a competitive advantage for the publisher as it
could increase the total number of submissions to the publisher's journals, including smaller and more specialized journals. Publishers 
may also recommend that authors publish their significant or very novel papers in more prestigious journals with higher publication 
charges (assuming an open access publisher that charges authors for publication).

\subsubsection*{Findings}

A group of 8 participants was presented with the second scenario of finding appropriate journals and adjusting the manuscript to the
journal's style. The participants discussed the potential competitive advantages of such a system using adaptive hybrid AI. The group 
argued that most benefits would be of type "Improved performance" as the system would help authors to find better suited journals and
hence reduce rejections. Further, For publishers, this would translate to less editorial work (reduced workload) for peer-reviewing
papers that finally get rejected. Further, as the system would format manuscripts according to the journal's style, the publisher has
less editing efforts. This was seen as a competitive advantage of both types, "Reduced costs" and "Improved performance". This 
finding also links back to the writing assistant in scenario 1, as the writing assistant could already format the manuscript according
to the journal's style.

\subsection{Editor Desk Review}

\subsubsection*{Adaptive and Hybrid AI Use Cases}

We now leave the author's perspective and move to the editor's perspectives. After the author has submitted the manuscript, an editor
will perform the initial desk review of the manuscript. The editor will check if the manuscript is within the scope of the journal, if
it is of sufficient quality, and formatted according to the journal's style. Desk review is an important step as it allows filtering out
unsuitable manuscripts early in the process. This saves time and resources for peer-reviewing and editing of manuscripts that will be
likely rejected anyway.

\subsubsection*{Competitive Advantages} 
\subsubsection*{Findings}


\subsection{Finding Peer-Reviewers}
\subsubsection*{Adaptive and Hybrid AI Use Cases}
\subsubsection*{Competitive Advantages} 
\subsubsection*{Findings}


\begin{landscape}
    \begin{table}[htb]
        \caption{
            Typical editorial processing steps for a journal manuscript and use cases for adaptive hybrid AI.
        }
        \label{tab:editorialProcess}

        \tiny
        \renewcommand{\arraystretch}{1.1}
        \small\centering
        \setlength\tabcolsep{8pt}
        \begin{tabularx}{\linewidth}{L{0.05} L{0.2} L{0.35} L{0.35}}
            \toprule
            \textbf{Step} & \textbf{Role / Task} & \textbf{AI Use Cases} & \textbf{Adaptability Aspect} \\
            \midrule

            \circled{1} & Author writes manuscript & 
                Literature search, literature recommendation, summarization of key findings, writing (auto-completion), translating, grammar and spell-checking &
                AI adapts to user by learning what papers are relevant, recommends more similar papers, and expands to related concepts. \linebreak
                AI suggests auto-completions and corrections based on the \textit{user's writing style} and the context of the manuscript.
                \\
            \midrule

            \circled{2} & Author searches journal to submit to & Journal recommendation &
                AI adapts to user by learning what journals are relevant to the user (journals read, cited, previously published in)
                and recommends more similar journals.
                \\
            \midrule

            \circled{3} & Author formats paper to meet journal requirements & Manuscript conversion and styling & 
                AI learns styles of journals and adapts its output to the journal selected by user \\
            \midrule

            \circled{4} & Author submits paper & Extraction of metadata & (Adaptability not required) \\
            \midrule
            
            \circled{5} & Editor receives submission & Summarization of key findings & (Adaptability not required) \\
            \midrule
            
            \circled{6} & Editor conducts desk review & Checks of the manuscript: detecting plagiarism, tortured phrases,
                generated papers, bias, inappropriate language, off-topic references, manipulated images, inappropriate authorship, etc.
                & AI needs to adapt to recent literature (e.g., plagiarism check needs
                to account for latest literature) and detecting new methods of generating papers.\\ 
            \midrule

            \circled{7} & Editor searches potential reviewers & Semantic search (in vector space using word embeddings), graph embeddings,
                review assignment algorithms using e.g., knowledge graph to exclude potential reviewers with conflicts of interest & AI needs to
                adapt to recent literature and previous reviewer preferences of the editor \\ 
            \midrule

            \circled{8} & Editor invites potential reviewers & E-mail writing, generate summary of the manuscript &
                AI should learn the user's writing style and typical wording from previous examples. \\
            \midrule

            \circled{9} & Reviewer reads manuscript & Summarization of key findings, checking of the content of cited references &
                AI needs to adapt to recent cited literature. \\ 
            \midrule

            \circled{10} & Reviewer writes review report & Writing (auto-completion) of qualitative review report: help reviewer to avoid biases,
                inappropriate feedback, lack of specificity & (Adaptability not required) \\ 
            \midrule

            \circled{11} & Editor reads review reports & Checking of the quality of the peer-review report: detect biases, inappropriate 
                language, generic / non-specific feedback, ad hominem criticism, off-topic comments, etc.  & (Adaptability not required) \\ 
            \midrule

            \circled{12} & Editor makes decision & Summarization of peer-review outcome for decision letter
                to author & AI should learn the user's writing style and typical wording from previous examples. \\ 
            \midrule

            \circled{13} & Author revises manuscript & Checking that reviewer concerns are being addressed, writing (auto-completion) of
                rebuttal letter to the reviewers \& editors & AI should learn the user's writing style and typical wording from previous examples. \\
            \bottomrule
        \end{tabularx}
    \end{table}
\end{landscape}


