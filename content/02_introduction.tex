\section{Introduction}
\label{sec:introduction}

As enterprises face universally accelerating change \citep{eliazarUniversalityAcceleratingChange2018}, it
is crucial for their artificial intelligence (AI) systems to be dynamic and adaptable. Adaptive and hybrid
intelligent (HI) systems that learn from new data, from the interaction with human agents and work
synergistically with humans can provide a competitive advantage to enterprises. This study explored the
competitive advantage for enterprises adopting adaptive hybrid intelligent systems using the case of a
typical editorial process in the scholarly publishing industry. The study answers the following research
question: \textit{How can AI systems learn from and adapt to humans and their environment for the
competitive advantage of enterprises?}

A mixed-methods approach was used, consisting of a literature review and qualitative data collection from
a focus group of graduate students in information systems ($n = 25$). The literature review identified how 
adaptive hybrid intelligent systems contribute to competitive advantages and was used to derive hypotheses
using the case of the scholarly publishing industry. The hypotheses were tested through qualitative feedback
from the focus group.

{\color{purple} @todo: after the workshop, add a short summary of the main hypotheses, findings, 
discussions and conclusion\dots}

The remainder of the paper is structured as follows. Section \ref{sec:literature} presents a literature review
on hybrid intelligent systems and their competitive advantage for enterprises. Section \ref{sec:methods} describes
the methodological approach of the study. Section \ref{sec:results} presents the findings. The paper concludes in
Section \ref{sec:discussion} with a discussion of the findings and limitations of the study.