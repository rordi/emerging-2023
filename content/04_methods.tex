\section{Methodology}
\label{sec:methods}

The study aimed to investigate the competitive advantage that can arise for an enterprise
through the adoption of hybrid intelligent systems. Specifically, the study explored the aspect of adaptability
of such hybrid intelligent systems. The study used a mixed-methods approach consisting of a literature
review (secondary data) and qualitative data collection from a focus group of 25 graduate students in the
FHNW Business Information Systems master program (primary data).

The literature review was conducted to identify factors that contribute to the competitive advantage
of enterprises using AI systems in general, and adaptable hybrid intelligent systems in particular.
The literature search was mainly conducted on Elicit \footnote{\href{https://elicit.org/}{elicit.org}}
and Google Scholar \footnote{\href{https://scholar.google.com/}{scholar.google.com}} using different
query terms, including ``competitive advantage of AI'', ``hybrid intelligent system'',
``expert system'', ``decision support system'', ``human-in-the-loop'', ``competitive advantage and AI'',
etc. Additionally, a forward and backward search was applied on relevant papers that were identified
from the initial literature searches.

The findings from the literature review were used to establish hypotheses on the competitive advantage
of adaptable hybrid intelligent systems for enterprises using the example of one industry. Given the
background knowledge of the author, the hypotheses were applied to the scholarly publishing industry.
To test the derived hypotheses, a focus group of students ($n = 25$) was selected based on their educational
background in business information systems. As part of a workshop the focus group was presented with the
hypotheses and asked to discuss and provided qualitative feedback for each hypothesis. Participants were
encouraged to provide detailed feedback on their experiences and perceptions related to the application
of the hypotheses in the industry case. The qualitative data was analyzed using thematic analysis and
common themes identified.