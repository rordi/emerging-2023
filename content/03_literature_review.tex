\section{Literature Review}
\label{sec:literature}

In this section we inform a background on different types of artificial intelligence (AI), hybrid intelligence, 
design patterns and principles for hybrid intelligent systems, and the competitive advantage arising from
the use of such systems in enterprises.

\subsection{Hybrid AI}

AI involves the creation of computer programs and algorithms that allow machines to replicate human cognition and behavior,
which includes the capabilities of perception, learning, reasoning, solving problems, and making decisions \citep{russel2010}.
AI can be broadly subdivided into symbolic and sub-symbolic approaches, see e.g., \cite{eliasmithSymbolicSubsymbolic2006}.
Symbolic approaches involve the use of explicit symbols and rules to represent knowledge and reason in a way that is easily
understood and explainable by humans, while sub-symbolic (or \textit{connectionist}) approaches aim to learn complex patterns
from vast amounts of data using neural networks \citep{ilkouSymbolicVsSubsymbolic2020}. Hybrid AI refers to systems combining
symbolic and sub-symbolic approaches. Hybrid AI systems can be anywhere from loosely coupled to tightly integrated
\citep{garcezNeurosymbolicAI3rd2023}.

Loosely coupled hybrid AI systems typically involve a human, which is also known as \textit{human in the loop} (HITL)
computing. In such systems the humans and AI work together towards common goals, augmenting the human intellect and
overcoming human limitations and cognitive biases \citep{akataResearchAgendaHybrid2020}. Hybrid AI systems show the
potential for improving the outcomes of AI systems, hence \textit{augmenting} rather than replacing human intelligence
\citep{akataResearchAgendaHybrid2020}. Hybrid approaches to AI have recently received attention from the leaders in
the field as a response to an ever increased focus on sub-symbolic approaches and deep learning in particular. In his
presidential address to the members of the Association for the Advancement of Artificial Intelligence (AAAI),
\cite{kambhampatiChallengesHumanAwareAI2020} has demanded that AI researchers build human-aware AI systems that work
synergistically with humans, including considering the human mental state, recognizing desires and intentions, and
providing proactive support to humans. In particular, AI researchers should aim at systems that show the capabilities
of \textit{explicability} (AI agents should show behavior that is expected by humans) and \textit{explainability}
(AI agents -- if behaving unexpectedly -- should be able to provide an explanation) \citep{kambhampatiChallengesHumanAwareAI2020}.

Such human-aware AI systems act as a human collaborator and must \textit{"sense, understand, and react to a wide
range of complex human behavioral qualities, like attention, motivation, emotion, creativity, planning, or argumentation"}
\citep{kortelingHumanArtificialIntelligence2021}. \cite{kortelingHumanArtificialIntelligence2021} further argue that the
pursuit of human-level intelligence is the wrong approach to AI. Due to the dissimilar nature of human and machine
intelligence, AI systems would need at some point to be \textit{degraded}: the physical substrate (biological,
respectively digital) determines the cognitive abilities and limitations of human \textit{versus} artificial
intelligence, with human cognitive faculties being limited by the biological and evolutionary origin of intelligence
\citep{kortelingHumanArtificialIntelligence2021}. They conclude that AI systems that support human decision-making
appear to be the best way forward for implementing better solutions, even if this means that we stick to narrow AI
 for the foreseeable future \citep{kortelingHumanArtificialIntelligence2021}.

Narrow AI refers to AI applications that have been trained with specific data for narrowly defined use cases,
typically yielding good performances on a single, predefined task. Hence, narrow AI applications typically lack
versatility: due to the limited amount and variety of training data, changing the use case of the AI typically
requires re-training a new model with different training data. On the other side, narrow AI application require fewer
data points and compute time for training compared to broad AI and may thus be re-trained more frequently or continuously
trained on new data (i.e., online training). Further, narrow AI models have a smaller number of parameters (i.e., weights,
biases) and thus also require less compute time and resources at inference time. In contrast, broad AI applications such
as foundation models are sophisticated and adaptive systems that successfully perform different cognitive tasks by virtue
of their sensory perception, computational learning, and previous experience \citep{hochreiterBroadAI2022}. 

Neurosymbolic AI -- a tightly integrated hybrid
AI approach -- is a promising approach to reach broad AI as it may eventually overcome the limitations of deep learning, such
as lack of explainability, susceptibility to adversarial attacks (data poisoning), and high computational cost
\citep{hochreiterBroadAI2022,garcezNeurosymbolicAI3rd2023}.


\subsection{Adaptive AI}

Adaptive AI: AI can adapt to user's specific needs. This can be part of a recommender system or a decision support
system that adapts according to the input received by users. As such, hybrid AI systems that are loosely coupled can 
be seen as adaptive AI systems. On the other hand, adaptive AI can also mean that the AI adapts to changes in the
environment, be it new data (adapting the AI to new data, i.e., retraining with new or more data or continuous online
learning of the AI model) or new use cases (adapting the task of the AI model).

\subsubsection{Foundation Models}

\begin{itemize}
    \item Language Models (LMs)
    \item neural networks trained of vast amounts of data, including on multimodal data (text, images, speech, video)
    \item Good at a variety of tasks, often the performance of LMs is closed to that of specialized model
    \item However, they show several limitations in their capabilities in reasoning and information retrieval. 
    \item Thus the terms "Foundation Model" was proposed by researchers at the Human-centered AI (HAI) institute of Stanford University
\end{itemize}

\begin{itemize}
    \item Foundation models represent a paradigm shift in AI
    \item ChatGPT as a chatbot is highly interactive: user has to prompt AI (althoug it is an unexplainable black box)
    \item Emerging capability in foundation models: in-context learning 
    \item In-context learning is highly adaptable: AI can learn from examples in the prompt 
\end{itemize}

\subsubsection{Limitations of Foundation Models}

\subsubsection{Hybrid Intelligent Approaches Involving Foundation Models}

\begin{itemize}
    \item Agents 
    \item Mixed architecture, e.g., MRKL
    \item Using the model as IR agent 
\end{itemize}


\subsection{Design Principles for Hybrid Intelligent Systems}

Hybrid AI systems can be represented by a boxology notation with common design patterns \citep{harmelenBoxologyDesignPatterns2019,
vanbekkumModularDesignPatterns2021,witschelVisualizationPatternsHybrid2021}.

\cite{ostheimerAllianceHumansMachines2021} developed a framework of eight principles for the design of human-in-the-loop (HITL) 
computing. They argue that such hybrid systems achieve higher accuracy and reliability of machine learning algorithms. Using a 
case in the manufacturing industry, they showed that the efficiency of operational processes could be increased by applying an 
algorithm that followed these design principles \citep{ostheimerAllianceHumansMachines2021}.

% box with HITL design principles
{
    \begin{center}
        \vskip 0.2in
        \fcolorbox{gray}{white}{
        \parbox{0.85\textwidth}{
        \textbf{Box 1. HITL Computing Design principles \citep{ostheimerAllianceHumansMachines2021}.}
        \begin{enumerate}
            \item Principle of client-designer relationship: designers should aim for mutual knowledge exchange with clients to foster the understanding
                    of which aspects of a system are influenced by human or artificial intelligence.
            \item Principle of sustainable design: designers should keep up to date with the latest progress in the field of AI and apply the latest and
                    lasting AI techniques.
            \item Principle of extended vision
            \item Principle of AI-readiness
            \item Principle of hybrid intelligence
            \item Principle of use-case marketing
            \item Principle of power relationship
            \item Principle of human-AI trust
        \end{enumerate}}}
        \vskip 0.2in
    \end{center}
}


\subsection{Types of Hybrid Intelligent Systems}

\begin{itemize}
    \item Expert systems 
    \item Decision support systems
    \item Recommender algorithms with human decision-making
    \item Case-based reason systems
\end{itemize}

\subsection{Enterprise Competitiveness}

{\color{purple} @todo: what are the aspects of and factors increasing the competitiveness of enterprises?}


\subsection{Competitive Advantage Through AI}

\cite{xuCanArtificialIntelligence2021} found that post COVID-19 companies using AI in their products grew
faster than their peers. However, they could not observe evidence of the same effect before COVID-19, indicating
that this development is either very recent or was fueled by the COVID crisis. More recently
\cite{hoArtificialIntelligenceFirm2022} reviewed the potential benefits of AI for enterprises as reported
by selected previous studies published between 2016 and 2021:

\begin{itemize}
    \item reduced costs
    \item improved performance
    \item better decision-making
    \item higher customer satisfaction
    \item better customer segmentation
    \item improved customer experience
    \item better products \& services
    \item business innovation
\end{itemize}

Further, \cite{hoArtificialIntelligenceFirm2022} identified several empirical studies that reported a positive,
neutral or negative effect of AI on enterprise performance. In particular one study by ...liu et al. (2022)...
and cited in \cite{hoArtificialIntelligenceFirm2022} reported negative performance of AI-related adoption 
announcements on firm market value for 62 listed US companies between 2015-2019.
