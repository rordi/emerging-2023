\section{Literature Review}
\label{sec:literature}

According to \cite{russel2010} artificial intelligence (AI) is the creation of computer programs and algorithms
that allow machines to replicate human cognition and behavior, which includes the capabilities of learning, perception,
reasoning, solving problems, and making decisions. AI can be broadly subdivided into symbolic and sub-symbolic approaches
\citep{eliasmithSymbolicSubsymbolic2006}. Symbolic approaches involve the use of explicit symbols and rules to 
represent knowledge and reason in a way that is easily understood and explainable by humans, while sub-symbolic
approaches aim to learn complex patterns from vast amounts of data using neural networks
\citep{ilkouSymbolicVsSubsymbolic2020}. Hybrid AI refers to systems involving symbolic and sub-symbolic approaches.
Applications of hybrid AI systems typically involve a human, which is also known as human-in-the-loop (HITL). 
More generally,  hybrid intelligence (HI) refers to systems where humans and AI systems work together towards common goals,
augmenting the human intellect and overcoming human cognitive biases \citep{akataResearchAgendaHybrid2020}.
Such systems show the  potential for improving the outcomes of AI systems, hence augmenting rather than replacing
human intelligence \cite[p. 19]{akataResearchAgendaHybrid2020}.

In practice, hybrid intelligent system often involve a mix of symbolic (i.e., knowledge representation) and
sub-symbolic (i.e., machine learning) approaches and can be represented by a boxology notation
\citep{harmelenBoxologyDesignPatterns2019,vanbekkumModularDesignPatterns2021}.
\cite{witschelVisualizationPatternsHybrid2021} extended the boxology notation to account for HITL in hybrid
intelligent systems.




\subsection{Categorization of Artificial Intelligence}

\begin{itemize}
    \item Symbolic versus sub-symbolic Artificial Intelligence
    \item Narrow artificial intelligence
    \item Artificial general intelligence (AGI)
    \item Hybrid intelligence (HI)
\end{itemize}





\subsection{Hybrid Intelligence}



\subsection{Design Principles Hybrid Intelligent Systems}

\cite{ostheimerAllianceHumansMachines2021} developed a framework of eight principles for the design of human-in-the-loop (HITL) 
computing. They argue that such hybrid systems achieve higher accuracy and reliability of machine learning algorithms. Using a 
case in the manufacturing industry, they showed that the efficiency of operational processes could be increased by applying an 
algorithm that followed these design principles \citep{ostheimerAllianceHumansMachines2021}.

% box with HITL design principles
{
    \begin{center}
        \vskip 0.2in
        \fcolorbox{gray}{white}{
        \parbox{0.85\textwidth}{
        \textbf{Box 1. HITL Computing Design principles \citep{ostheimerAllianceHumansMachines2021}.}
        \begin{enumerate}
            \item Principle of client-designer relationship: designers should aim for mutual knowledge exchange with clients to foster the understanding
                    of which aspects of a system are influenced by human or artificial intelligence.
            \item Principle of sustainable design: designers should keep up to date with the latest progress in the field of AI and apply the latest and
                    lasting AI techniques.
            \item Principle of extended vision
            \item Principle of AI-readiness
            \item Principle of hybrid intelligence
            \item Principle of use-case marketing
            \item Principle of power relationship
            \item Principle of human-AI trust
        \end{enumerate}}}
        \vskip 0.2in
    \end{center}
}


\subsection{Types of Hybrid Intelligent Systems}

\begin{itemize}
    \item Expert systems 
    \item Decision support systems
    \item Recommender algorithms with human decision-making
\end{itemize}


\subsection{Foundation Models}

\begin{itemize}
    \item Language Models (LMs)
    \item neural networks trained of vast amounts of data, including on multimodal data (text, images, speech, video)
    \item Good at a variety of tasks, often the performance of LMs is closed to that of specialized model
    \item However, they show several limitations in their capabilities in reasoning and information retrieval. 
    \item Thus the terms "Foundation Model" was proposed by researchers at the Human-centered AI (HAI) institute of Stanford University
\end{itemize}

\begin{itemize}
    \item Foundation models represent a paradigm shift in AI
    \item ChatGPT as a chatbot is highly interactive: user has to prompt AI (althoug it is an unexplainable black box)
    \item Emerging capability in foundation models: in-context learning 
    \item In-context learning is highly adaptable: AI can learn from examples in the prompt 
\end{itemize}

\subsubsection{Limitations of Foundation Models}

\subsubsection{Hybrid Intelligent Approaches Involving Foundation Models}

\begin{itemize}
    \item Agents 
    \item Mixed architecture, e.g., MRKL
    \item Using the model as IR agent 
\end{itemize}





\subsection{Enterprise Competitiveness}

{\color{purple} @todo: what are the aspects of and factors increasing the competitiveness of enterprises?}


\subsection{Competitive Advantage Through Adaptive Hybrid Intelligent Systems}

\cite{xuCanArtificialIntelligence2021} found that post COVID-19 companies using AI in their products grew
faster than their peers. However, they could not observe evidence of the same effect before COVID-19, indicating
that this development is either very recent or was fueled by the COVID crisis. More recently
\cite{hoArtificialIntelligenceFirm2022} reviewed the potential benefits of AI for enterprises as reported
by selected previous studies published between 2016 and 2021:

\begin{itemize}
    \item reduced costs
    \item improved performance
    \item better decision-making
    \item higher customer satisfaction
    \item better customer segmentation
    \item improved customer experience
    \item better products \& services
    \item business innovation
\end{itemize}

Further, \cite{hoArtificialIntelligenceFirm2022} identified several empirical studies that reported a positive,
neutral or negative effect of AI on enterprise performance. In particular one study by ...liu et al. (2022)...
and cited in \cite{hoArtificialIntelligenceFirm2022} reported negative performance of AI-related adoption 
announcements on firm market value for 62 listed US companies between 2015-2019.
